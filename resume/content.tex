%====================
% Objective Statement
%====================

\begin{center}
\textbf{Electrical engineer and research coordinator with experience in hardware design for high-speed digital communication, photonic systems, and academic research processes for geophysical science applications.}
\end{center}
\setlength{\parindent}{1.5em}
\small My background includes avionics sensor design at Amazon, signal integrity engineering at Microsoft, and academic research coordination for geophysics and environmental sensing groups at the University of Washington. I am interested in experimental design workflows, precision measurement systems, and field operations for high-quality scientific data collection. I am actively seeking roles in applied research and instrumentation alongside future graduate study at the intersection of hardware engineering, experimental physics, and geophysical sciences.
\vspace{4pt}

%====================
% EDUCATION
%====================

\section{Education}

\subsection{{Bachelor of Science in Electrical Engineering \hfill August 2015 – May 2019}}
\subtext{University of Illinois at Urbana–Champaign \hfill Champaign-Urbana, IL}
\begin{zitemize}
    \item Relevant coursework: Electromagnetics, Signal Processing, Semiconductor Physics, Analog and Digital Circuits, IC Theory \& Fabrication.
    \item Internships: GE Global Research (Summer 2016 \& 2017), Amazon (Summer 2018).
    \item Exchange Semester: Technical University of Denmark (Autumn 2017).
\end{zitemize}

%====================
% TECHNICAL EXPERIENCE
%====================

\section{Technical Experience}

\subsection{{Research Coordinator (Photonics) \hfill May 2024 – May 2025}}
\subtext{University of Washington, College of the Environment, Department of Earth \& Space Sciences \hfill Seattle, WA}
\begin{zitemize}
    \item Supported deployment, operation, testing, and maintenance of photonics-based Distributed Acoustic Sensing (DAS) data acquisition systems for seismic, oceanic, and cryospheric research.
    \item Coordinated technical activities with external research partners and funding agencies; acted as operational liaison between research groups and departmental administration by managing procurement, inventory, shipping, and customs documentation.
    \item Developed and maintained experimental system workflows for large-volume data management, metadata documentation, and technical reporting across multi-institution collaborations; maintained project websites and supported data dissemination platforms.
    \item Supported safety and logistical planning and permitting for remote field deployments and instrumentation campaigns.
\end{zitemize}

\subsection{{Hardware Engineer II (Signal Integrity) \hfill September 2021 – September 2023}}
\subtext{Microsoft, Cloud AI Hardware and Advanced Signal Engineering \hfill Seattle, WA}
\begin{zitemize}
    \item Modeled and analyzed end-to-end signal integrity performance across FPGA die, packages, connectors, and board-level interconnects.
    \item Conducted high-speed hardware validation and SI characterization for enterprise AI and cloud infrastructure platforms.
    \item Performed electromagnetic and circuit-level simulations using tools such as ANSYS HFSS and Keysight ADS.
    \item Supported board-level stackup design, routing constraints, and interference mitigation strategies to resolve system integration issues.
    \item Executed laboratory measurements, including S-parameters and eye-diagram analysis, using VNA and time-domain methods.
    \item Supported contract PCB design reviews and monitored progress through manufacturing readiness.
\end{zitemize}

\subsection{{Hardware Design Engineer (Satellite Avionics) \hfill April 2020 – September 2021}}
\subtext{Amazon, Project Kuiper \hfill Seattle, WA}
\begin{zitemize}
    \item Supported early-phase (pre-PDR) satellite avionics sensor subsystem design and development.
    \item Authored design review documents and test validation for electrical, thermal, and mechanical performance.
    \item Defined early avionics sensor requirements from component to system level.
    \item Designed and reviewed schematics, PCB layouts, and housings for sensor prototypes.
    \item Performed board bring-up, debugging, and validation.
    \item Supported hardware radiation test campaigns.
\end{zitemize}

\subsection{{Technical Program Manager (Robotic Automation Prototypes) \hfill June 2019 – March 2020}}
\subtext{Amazon, Worldwide Technical Engineering Services \hfill Seattle, WA}
\begin{zitemize}
    \item Supported nationwide deployment of PLC-based robotic automation prototypes across fulfillment centers.
    \item Managed schedules, logistics, and readiness planning for pilot programs.
    \item Coordinated on-site installations, commissioning, and documentation with engineering and operations teams.
    \item Tracked development milestones through production and operational handoff.
    \item Supported standardization of risk reviews and vendor evaluations.
\end{zitemize}

%====================
% SKILLS
%====================

\section{Skills}

\renewcommand{\arraystretch}{1.2}

\begin{tabularx}{\linewidth}{p{3.2cm} X}

\skills{Communication} &
\textbf{English} (native), Spanish (A2), French (beginner) \\

\skills{Programming} &
Python, C/C++, MATLAB, JavaScript, LaTeX, Markdown \\

\skills{Development} &
Git, Jupyter, Linux (Bash/Shell), Microcontrollers, ArcGIS, Docker, HW/SW RAID storage, HPC remote systems \\

\skills{Instrumentation} &
Distributed Acoustic Sensing (DAS), Data Acquisition Systems (DAQ), Noise Characterization, Sensor Calibration, Signal Integrity Analysis, Channel Simulation, Schematic Design, PCB Layout, Cyclotron Hardware Radiation Effects Testing \\

\skills{Field Operations} &
Hands-on system deployment, field logistics coordination, remote system monitoring, experimental setup and testing, test planning, data management, technical documentation, American Mountain Guides Association (AMGA) Professional Member \\

\end{tabularx}

%====================
% ACTIVITIES
%====================

\section{Activities}

\subsection{{Field \& Alpine Activities}}
\begin{zitemize}
    \item Glacier travel, alpine climbing, skiing, and cross-training; prospective AMGA Alpine Guide.
    \item Experience planning safety, navigation, logistics, and remote system management for field exploration in diverse environments.
\end{zitemize}
\vspace{-0.5em}
\begin{multicols}{2}[
\raggedcolumns
]
\noindent\textbf{Certifications}
\vspace{-0.5em}
\begin{zitemize}
    \item Wilderness First Responder (WFR) with AED and CPR Certification, Yosemite, CA (November 2025)
    \item AIARE 1 Avalanche Training Certification, Skyward Mountaineering, Silverton, CO (December 2025)
    \item Avalanche Companion Rescue Training, Alpine Ascents International x SheJumps, Snoqualmie, WA (January 2026)
\end{zitemize}
\columnbreak
\noindent\textbf{Professional Affiliations}
\vspace{-0.5em}
\begin{zitemize}
    \item American Mountain Guides Association (AMGA) Professional Member
    \item American Alpine Club (AAC) Member
    \item Boeing Alpine Society (BOEALPS) Member
    \item SnowGoat Skimo Volunteer
    \item Washington State Rare Plant Monitor
\end{zitemize}
\end{multicols}

\subsection{{Conferences \& Workshops}}
\begin{zitemize}
    \item BOAT Ocean Acoustics Workshop, University of Washington (2025)
    Two-day intensive workshop on experimental signal processing theory.
    \item DesignCon, Santa Clara, CA (2022)
    Industry conference on high-speed signal integrity measurement and PCB manufacturing.
\end{zitemize}